
% Created 2018-08-01 Wed 12:57
% Intended LaTeX compiler: pdflatex
\documentclass{article}
               \usepackage{listings}
               \usepackage{color}
               \usepackage{tocloft}

               \usepackage{times}
%\usepackage[cmbold]{mathtime}
\usepackage{bm}
\usepackage{natbib}


% --------- OWN START

\usepackage{hyphenat}
\usepackage{bbold}
\usepackage{bbm}
\usepackage[T1]{fontenc}
\usepackage{amsmath}
\usepackage{amssymb}
\usepackage{prodint}
\usepackage{color}

               
\usepackage{amsmath}
\usepackage{array}
\usepackage[T1]{fontenc}
\usepackage{natbib}
\bibliographystyle{abbrvnat}
\setcitestyle{authoryear,open={(},close={)}}
\usepackage{authblk} 
\usepackage{mathrsfs}
\usepackage{enumitem}
\newcommand*{\qeda}{\hfill\ensuremath{\tiny\CIRCLE}} 
\usepackage[small]{titlesec}
% \titleformat{\subsection}%[runin]
% {\large\bfseries}{\thesubsection}{1em}{}
\titleformat{\subsubsection}[runin]
{\normalfont\bfseries}{\thesubsubsection}{1em}{}
%{\normalfont\itshape}{\normalfont\thesubsubsection}{1em}{}
\usepackage{adjustbox}
\usepackage{prodint}
\usepackage{booktabs}
\usepackage{bm}
\usepackage{bbold}
%\setlength\parindent{0pt}
\usepackage{wasysym}
\usepackage[margin=1.3in]{geometry}
\setcounter{secnumdepth}{4}

\newcommand{\cadlag}{c\`{a}dl\`{a}g }
\newcommand{\Cadlag}{C\`{a}dl\`{a}g }
\newcommand{\EE}{\mathbb{E}}
\newcommand{\one}{1}
\newcommand{\eps}{\varepsilon}
\newcommand{\VV}{V}
\newcommand{\PP}{\mbox{P}}
\newcommand{\norm}{\mathcal{N}}
\newcommand{\lag}{N}
\newcommand{\str}{S}
\newcommand{\smin}{s^{\min}}
\newcommand{\smax}{s^{\max}}
\newcommand{\styp}{s^{*}}
\newcommand{\period}{[a,b]}
\newcommand{\periodK}{\ensuremath{[T_k,T_{k+1})}}
\newcommand{\K}{K}
\newcommand{\kk}{k}
\newcommand{\D}{D}
\newcommand{\B}{B}
\newcommand{\F}{\mathcal{F}}
\newcommand{\E}{E}
\newcommand{\XX}{X}
\newcommand{\QQ}{Q}
\newcommand{\Ru}{R}
\newcommand{\GG}{G}
\newcommand{\T}{T}
\newcommand{\st}{s}
\newcommand{\Nn}{N}
\newcommand{\C}{C}
\newcommand{\uu}{u}
\newcommand{\vv}{v}
\newcommand{\zz}{z}
\newcommand{\ww}{w}
\newcommand{\M}{M}
\newcommand{\I}{I}
\newcommand{\RR}{R}
\newcommand{\R}{\mathbb{R}}
\newcommand{\N}{\mathbb{N}}
\newcommand{\1}{\mathbb{1}}
\newcommand{\ses}{SES}
\newcommand{\Y}{Y}
\newcommand{\htt}{h}
\newcommand{\dtt}{d}
\newcommand{\tM}{M}
\newcommand{\ta}{\tau_0}
\newcommand{\tb}{\tau_1}
\newcommand{\intd}{\mathcal{D}}
\newcommand{\nW}{n_W}
\newcommand{\nL}{n_L}
\newcommand{\tmax}{\tau}
\newcommand{\A}{A}
\newcommand{\LL}{L}
\newcommand{\W}{W}
\newcommand{\X}{X}
\newcommand{\U}{U}
\newcommand{\Z}{Z}
\newcommand{\btheta}{\bm{\theta}}
\newcommand{\Pvar}{W}
\newcommand{\pvar}{p}
\newcommand{\logit}{\text{logit}}
\newcommand{\expit}{\text{expit}}
\newcommand{\argmax}{\text{argmax}}
\newcommand{\argmin}{\text{argmin}}
\newcommand{\cupdot}{\mathbin{\mathaccent\cdot\cup}}
\newcommand{\qed}{\hfill\(\square\)}
\newcommand\independent{\protect\mathpalette{\protect\independenT}{\perp}}\def\independenT#1#2{\mathrel{\rlap{$#1#2$}\mkern2mu{#1#2}}}
\newtheorem{thm}{Theorem}[section]
\newtheorem{claim}{Claim}[section]
\newtheorem{defi}{Definition}
\newtheorem{lemma}[thm]{Lemma}
\newtheorem{example}[thm]{Example}
\newtheorem{obs}{Observation}
\newtheorem{drug}{Drug type}
\renewcommand\thedrug{\Roman{drug}}
\newtheorem{remark}{Remark}
\newtheorem{assumption}{Assumption}
\newtheorem{inclusion}{Inclusion criterion}
\newtheorem{proxy}{Proxy}
\usepackage{tikz}
\usetikzlibrary{arrows}
\tikzset{every picture/.style=remember picture}
\newcommand{\mathnode}[1]{\mathord{\tikz[baseline=(#1.base), inner sep = 0pt]{\node (#1) {$#1$};}}}

\lstset{
keywordstyle=\color{blue},
commentstyle=\color{red},stringstyle=\color[rgb]{0,.5,0},
literate={~}{$\sim$}{1},
basicstyle=\ttfamily\small,
columns=fullflexible,
breaklines=true,
breakatwhitespace=false,
numbers=left,
numberstyle=\ttfamily\tiny\color{gray},
stepnumber=1,
numbersep=10pt,
backgroundcolor=\color{white},
tabsize=4,
keepspaces=true,
showspaces=false,
showstringspaces=false,
xleftmargin=.23in,
frame=single,
basewidth={0.5em,0.4em},
}

\usepackage[utf8]{inputenc}
\usepackage[T1]{fontenc}
\usepackage{graphicx}
\usepackage{grffile}
\usepackage{longtable}
\usepackage{wrapfig}
\usepackage{rotating}
\usepackage[normalem]{ulem}
\usepackage{amsmath}
\usepackage{textcomp}
\usepackage{amssymb}
\usepackage{capt-of}
\usepackage{hyperref}


\title{Stochastic intervention: Brystsmerter and ambulance}
\author{Helene C. W. Rytgaard}

\begin{document}

\maketitle


% \section*{Supplementary material}

\section{Notation}

We consider data as follows: \(X\in \R^d\) are general covariates,
\(W\in\lbrace 0,1\rbrace\) is an indicator of brystsmerter,
\(A\in \lbrace 0,1\rbrace\) is an indicator of ambulance,
\(Y\in\lbrace 0,1\rbrace\) is an outcome indicator (30 day survival).\\

Let \(\pi( W, X )= P(A=1 \mid  W, X)\) denote the conditional
distribution of ambulance. Let \(\hat{\pi}_n\) denote its estimator
(super learner).  \\

All data are used for initial estimation. \\

Note that the population in the following sections consists only of
those \(i\) for which \(W_i = 0\). Thus, \(n\) are the number of
subjects with \(W_i=0\).


\section{Target parameter}

Our target parameter is:
\begin{align*}
  \Psi(P)  = \Psi_1 (P) - \Psi_2 (P) ,
  %\qquad \text{for, } \, P \in \mathcal{M},
\end{align*}
where
\begin{align}
  \Psi_1 (P)  = \EE\bigg[ \sum_{a=0,1}\EE [Y \mid A=a, W=0, X]
  \hat{\pi}_n ( W=1, X) \bigg],
  \label{eq:psi:1}
\end{align}
with the outer expectation taken over the observed distribution of
\(X\), and,
\begin{align}
  \Psi_2 (P) = \EE\big[ \EE [Y \mid A, W=0, X]
  \big],
    \label{eq:psi:2}
\end{align}
with the outer expectation taken over both \(A\) and \(X\). Note that
\(\Psi_1\) defined by \eqref{eq:psi:1} is the risk among those with no
brystsmerter (\(W=0\)) had they had an ambulance sent for them with
the same probability that an ambulance was sent to someone in the same
population with same characteristics \(X\) but \textit{with}
brystsmerter.  \(\Psi_2\) defined by \eqref{eq:psi:2} is the observed
risk among those who did not report brystsmerter (\(W=0\)).

\subsection{Efficient influence function}

 The efficient influence function for \(\Psi_1\) is given
by
\begin{align*}
 & \phi_1 ( P ) (O) =  \frac{ \hat{\pi}_n ( W=1, X)^A \big( 1-
  \hat{\pi}_n ( W=1, X)\big)^{1-A}}{\pi ( W=0, X)^A \big( 1-
  \pi ( W=0, X)\big)^{1-A}} \big( Y - \EE[Y \mid A,W=0,X] \big)  \\
&\qquad\qquad\qquad\qquad\qquad  + \, 
  \sum_{a=0,1}\EE [Y \mid A=a, W=0, X]
  \hat{\pi}_n ( W=1, X) - \Psi_1(P)(O), 
\end{align*}
and the efficient influence function for \(\Psi_2\) is given by
\begin{align*}
  \phi_2 ( P )(O) = \big( Y - \EE[Y \mid A,W=0,X] \big)  +
  \EE [Y \mid  W=0, X]
  - \Psi_2(P)(O). 
\end{align*}
Thus, the efficient influence function for \(\Psi = \Psi_1 - \Psi_2\)
is
\begin{align*}
\phi (P) (O) = \phi_1(P)(O) - \phi_2(P)(O). 
\end{align*}

\section{Asymptotic distribution of estimator}

Under regularity conditions, bla bla, we can compute standard errors
for a TMLE estimator \(\hat{\psi}^*_n\) for \(\Psi(P_0)\) simply by:
\begin{align*}
  \sqrt{\frac{\hat{\sigma}_n}{n}},\qquad  \text{ where }  \quad \hat{\sigma}^2_n = \frac{1}{n} \sum_{i=1}^n
\big(  \phi (\hat{P}_n^*) (O_i)\big)^2 , 
\end{align*}
the latter being an estimate of the variance of the efficient
influence function. With estimators \(\hat{\pi}_n(w, X)\) for
\(P(A=1 \mid W=w, X)\) and \(\hat{Q}_n( a, L, w)\) for
\(Q ( a, L, w) = \EE [Y \mid A=a, W=w, L]\) the summand of the formula
above reads:
\begin{align*}
  & \phi (\hat{P}_n^*) (O_i)
    =   \frac{ \hat{\pi}_n ( W=1, X_i)^{A_i} \big( 1-
    \hat{\pi}_n ( W=1, X_i)\big)^{1-A_i}}{\hat{\pi}_n ( W=0, X_i)^{A_i} \big( 1-
    \hat{\pi}_n  ( W=0, X_i)\big)^{1-A_i}} \big( Y - \hat{Q}_n ( A, W=0, X_i)  \big)  \\
  &\qquad\qquad\qquad\qquad\qquad\quad\qquad  + \, 
    \sum_{a=0,1} \hat{Q}_n ( A=a, W=0, X_i)
    \hat{\pi}_n ( W=1, X_i) - \hat{\psi}_1  \\
  &\quad + \,   \big( Y_i -   \hat{Q}_n ( A_i, W=0, X_i) \big)  +
    \sum_{a=0,1} \hat{Q}_n ( A=a, W=0, X_i)
    \hat{\pi}_n ( W=0, X_i)
    -\hat{\psi}_2. 
\end{align*}


\newpage

\bibliographystyle{biometrika}
\bibliography{refs}




\end{document}












